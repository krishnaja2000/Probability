(i)To draw a histogram, the data must be made continuous.\\
\begin{align}
Gap/2=\frac{127-126}{2}=\frac{1}{2}=0.5\\
\end{align}
So we add 0.5 to every upperclass limit and subtract 0.5 from every lower class limit to obtain continous grouped frequency distribution table as shown in Table\ref{table:cont_stat_40}
\begin{table}[ht!]
\centering
\input{./stat/tables/exer40/cont40.tex}
\caption{Illness and fatality rate amongst women}
\label{table:cont_stat_40}
\end{table}
The histogram is shown in figure \ref{fig:hist40_py}
\begin{figure}[!ht]
\centering
\includegraphics[width=\columnwidth]{./codes/pyfigs/exer40.eps}
\caption{Length of leaves in mm}
\label{fig:hist40_py}
\end{figure}
The below python code was used o generate the histogram \ref{fig:hist40_py}
\begin{lstlisting}
./stat/codes/exer40.py
\end{lstlisting}
(ii)The same data can also be represented using a frequency polygon.\\
(iii)No, the maximum leaves have lengths ranging from 144.5mm to 153.5mm and might not be equal to 153mm.\\
