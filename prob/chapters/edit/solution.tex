(i)The sample size 
\begin{align}
S=90
\end{align}
(i)number of discs bearing a two digit number is 
\begin{align}
T=81
\end{align}
The probability of drawing a disc bearing two digit number is 
\begin{align}
\pr{T} = \frac{T}{S} &= \frac{81}{90}
\\
&= \frac{9}{10}
\end{align}

(ii)number of discs bearing a perfect square is 
\begin{align}
Sq=9
\end{align}
The probability of drawing a disc bearing perfect square is 
\begin{align}
\pr{Sq} = \frac{Sq}{S} &= \frac{9}{90}
\\
&= \frac{1}{10}
\end{align}

(iii)number of discs bearing number divisible by 5 is 
\begin{align}
F=18
\end{align}
The probability of drawing a disc bearing number divisible by 5 is 
\begin{align}
\pr{F} = \frac{F}{S} &= \frac{18}{90}
\\
&= \frac{1}{5}
\end{align}
The python code for the above solution is
\begin{lstlisting}
./prob/codes/exer129.py
\end{lstlisting}
